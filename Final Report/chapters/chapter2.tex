\chapter{Methods}
\label{chapter2}

\section{Version Control}
All of the code developed for this project has been managed by git and pushed to a github respository with can be accessed by the link:

\begin{center}
    https://github.com/Anthony-Moran/Final-Year-Project
\end{center}

The history will show the various stages of this project and how we started off with a blank workspace and ended up with this product now.
\linebreak
When implementing the online features, the commit messages became less detailed because the changes were extremely minor (usually one line or printing variables to the console) and changes were being made rapidly for debugging. This was required because the tools that we have used to deploy the app require the changes to be committed in order to come into effect. This differs from the earlier commits where the solution could first be tested locally and then committed only when the changes were implemeted correctly.

During this time, a new branch was created in order to isolate the changes, and ensure that there was a stable version of the app as a backup. Since then, the online version has been tested (more details in Chapter 3) and now works, so the main branch has now been pulled into this branch and the old one has since been deleted. A lot of the changes from this pull request actually happen outside of git itself. This includes setting up a server for the web app as well as configuring a service to host the websocket server. This configuration did include two new files that can be seen in the pull request, which is the Procfile as well as a requirements.txt file, which were necessary to make the websocket server work. Finally, the last major change was made to the server (main.py file) and this was so the backend could function correctly with the websocker server provider. However it should be noted that this change does not overwrite the offline version of the program, which can still be accessed by running the server on a user's devive (other users who wish to join must be on the same network due to the unsecure communication, more on that later).

\section{Back End}

The back end of a web app is often refererred to as the "brains" of a website/webapp. This part is responsible for all of the computation, storing of information and is where the server resides. The complement of this is the front end, which receives information from the server and displays the results in a way that is visually pleasing to the end user. These two components are tied together by some form of communication, and there are multiple ways this can be achieved.

% Processing logic on server side to reduce computation on client side and also means there is a single point of truth

The decision for how to distribute computation between server and client can be made by considering multiple factors. For example, computationally high tasks (like cloud gaming) are better suited to the back end because dedicated hardware can process information faster than the average computer or mobile device. However this speed can be capped by the user's bandwidth, which describes the volume of data they can receive over a given time, for example 80Mbps, or 80 megabits per second. Therefore if a calculation can be performed on the client side, that should be preferred so we can cut out the communication entirely. Now the app we are developing is not computationally expensive, so we should process data on the client side right? While this would work, it may not be the best solution for this probem because it would mean duplicating data, for the two players in the game and potentially other spectators. This is common in distirbuted system design and it certainly works but in this case it would be better to store all information on the server so that there is a single point of truth. This ensures that all participants are synced and if the client needs information, the server is responsible for sending it.

\subsection{Chess Logic}

Mention transition from old chess engine to python-chess. The old version was going to be used when the project started off as machine learning but in the end it wasn't fully implemented so now we are using the built in chess module.

Board is represented using FEN.

\subsection{Game Management}

Storing a tuple of game state and connected users in a dictionary with join key\linebreak
Generating join keys

\subsection{Python Server}

Offline server, explain how to enact

\subsection{Github Pages}

Online server

\section{Communication}

Network speed - applicable to online version\linebreak
Consider security\linebreak
Bad join keys redirect users to home menu

Challenege with matching the backend board with the frontend representation: orientation of pieces vs canvas coordinates

\subsection{Websockets}
Code was adapted from https://websockets.readthedocs.io/en/stable/\linebreak
Flawed with short life time and disconnect easily\linebreak
Pro is that they use bidirectional channel, so polling is no longer required\linebreak
Iphone will disconnect if locked/leaves safari (added cheap fix to allow reconnection but it's not special to the user, anyone can connect in their place)\linebreak
Refreshing with one user would cause the game to be deleted because the server would temporarily see that there are no users in the game. And after the page refreshed the user wouldn't be able to join because the game is now deleted. Added a small buffer time to keep the game alive, to combat this.\linebreak

Closed after win because no more messages need to be exchanged.

\subsection{Heroku}

Platform as a service (distirbuted systems?)

Server sleeps after a period of not being used. Causes next request to take a few seconds longer than usual


\section{Front End}

Important that the interface is graphical to make the softaware easier to adopt by others.

\subsection{The Welcome Page}

Option to start a new game\linebreak
Alternatively you can enter a code and join an existing game

Red message appears under user input if they entered an invalid code. Catches the eye and red subconsciously indicates error.

\subsection{The Chess Page}

Nav bar to return to main menu\linebreak
Text at the top to show join code and link\linebreak
Also prompts on who's turn it is and whether user is in check or if the game has finished with the result.\linebreak
board is drawn with canvas

reference sprite sheet\linebreak
requires javascript\linebreak
interaction through mouse click events : converts mouse position to row and column using modolos (check spelling) and integer division\linebreak
only parts of the board that need updating are updated: only drawing pieces that have moved or been taken, more resource efficient than redrawing the whole board.

\subsection{Responsive Design}

(CSS) Prioritises mobile devices and forces computers to adjust the layout. Computers typically have faster processors than mobile devices.\linebreak
computer users can resize the window and the design adapts: https://web.archive.org/web/20220714020647/https://bencentra.com/code/2015/02/27/optimizing-window-resize.html (debouncing?)