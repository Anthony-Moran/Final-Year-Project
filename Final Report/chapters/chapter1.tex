\chapter{Introduction and Background Research}

% You can cite chapters by using '\ref{chapter1}', where the label must
% match that given in the 'label' command, as on the next line.
\label{chapter1}

% Sections and sub-sections can be declared using \section and \subsection.
% There is also a \subsubsection, but consider carefully if you really need
% so many layers of section structure.
\section{Introduction}

Chess is a board game that has existed since the 7th century \cite{AHistoryOfChess}. The game is played by two people, each one in control of an army of equal strength; it is up to the player's logical reasoning and deduction in order to conquer the board. For a long time, chess could only be played in person, or perhaps through the post. Internet Chess Club was founded by Danny Sleator in 1992 \cite{InternetChessClub}, and he lead a small team of programmers to develop the first dedicated chess server. This was the introduction to playing chess over the internet and it allowed people to play chess together, regardless of the distance between them. It wasn't until 1995, where the first web based chess server was launched by Caissa \cite{Caissa}, which featured a graphical user interface. This most likely contributed to a higher adoption of playing chess online because it gave users a friendlier interface, which was more intuitive and approachable than what was previously available. Since then, many similar services have been created such as <list examples> and it goes to show that there is quite some variety in the way this service is implemented.


% Must provide evidence of a literature review. Use sections
% and subsections as they make sense for your project.
\section{Literature review}

\section{Libraries and Tools}

\subsection{Python}

Flask / Flask Restful\linebreak
Keep brief because it is not in the final solution\linebreak
Initially chosen due to comfortablility and experience\linebreak
Used polling

python modules here

\subsection{Web Languages: HTML, CSS and Javascript}

websockets

\subsection{Github Pages}

\subsection{Heroku}

\section{Background Experience}

Distributed Systems' - Webservice coursework and how it assisted in developing a web app
building for system vs building for user
