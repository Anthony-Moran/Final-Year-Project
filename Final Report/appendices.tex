\begin{appendices}

\chapter{Self-appraisal}

\section{Critical self-evaluation}

The biggest critique I would give myself is grossly underestimating the amount of time required to complete a project of this scale. I also prioritised the programming too much and gave myself less time to write the report. If I could go back, I would have stuck to the flaskRESTful implementation and the old chess engine because even though they were not the best tools for completing the task, the fact of the matter is that they did work (to a degree). If I had done this, instead of switching tools really late, I could have given myself adequate time to write the report and possibly write it to a higher standard I know I could achieve.

I also do not like the big code files I have ended up with. I am generally used to programming in a modular style, where code doesn't usually take up more than a page and a half per file, if that. I often found myself scrolling up and down for long periods of time trying to find specific sections of the code (in hindsight, I could have searched for it using command/ctrl F). However, with modular code this isn't a problem at all. An attempt was made with the network handler and the game logic and then towards the late half of development, I was starting to use for files for more specific tasks, like the confetti file or hover handler file.

Despite all of the negatives, I am very proud of the program that I have written. It is very polished and it can withstand a lot of extreme cases, especially in regards to websocket connections. Although on that note, I had only discovered socket.IO before it was too late and this provides an abstraction layer for websockets, which could have taken some of the headache out of developing them from scratch. However, having to get my hands dirty with websockets has given me a greater understanding of how they function and I will appreciate the abstraction more now in the future.

\section{Personal reflection and lessons learned}

On the bright side of things, I have learnt a new communication protocol, which I plan to use in future, real time games. I was also quite happy when I was able to discover polling organically by my own intuition (even though I didn't end up using it in the end).

I have also learnt that things can quickly become messy when storing everything on the server side. If I had made the code more modular I could have avoided some of the problems but not all of them. I think it may have been quite inefficient for the client to request the same piece of information repeatedly, when it could store the data itself and then update it when the server says to.

In the future I need to start projects much earlier to allow myself to divide the work into smaller, more manageable pieces. By doing this I can fully concentrate on one section at a time and ensure it is the best quality it can be before moving onto the next part.

\section{Legal, social, ethical and professional issues}

\subsection{Legal issues}

I do not believe there any legal issues with my project. There would be a potential in borrowing other people's work through the form of code or art but all cases of this have been referenced and they are available for public use.
Additionally, the content itself has no legal ties, it's a virtual board game.

\subsection{Social issues}

My program does face any social issues. Two people can only play a game with each other if they already know each other outside of the game because they need to communicate the code with each other. This removes the risk of online bullying, and both users are anonymous.

\subsection{Ethical issues}

Due to the nature of the project there are no ethical issues, all this program does is allow people to play chess together.

\subsection{Professional issues}

Similarly to the previous section, there are no professional issues in my project as it is a neutral environment.

\chapter{External Material}
% <This appendix should provide a brief record of materials used in the solution that are not the student's own work. Such materials might be pieces of codes made available from a research group/company or from the internet, datasets prepared by external users or any preliminary materials/drafts/notes provided by a supervisor. It should be clear what was used as ready-made components and what was developed as part of the project. This appendix should be included even if no external materials were used, in which case a statement to that effect is all that is required.>

\begin{itemize}
\item Code was taken from the debouncing section of this website: \url{https://web.archive.org/web/20220714020647/https://bencentra.com/code/2015/02/27/optimizing-window-resize.html}

\item The sprite sheet was taken from this website and is licensed under creative commons: \url{https://commons.wikimedia.org/wiki/Template:SVG_chess_pieces}

\item The structure of my server and client were heavily inspired by the code on this website, specifically listed under "Summary": \url{https://websockets.readthedocs.io/en/stable/intro/tutorial3.html}

\item Code was taken from this website to fix the issue with iPhone's inconsistent full height measurement: \url{https://dev.to/maciejtrzcinski/100vh-problem-with-ios-safari-3ge9}
\end{itemize}

\end{appendices}
