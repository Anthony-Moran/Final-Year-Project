\begin{appendices}

%
% The first appendix must be "Self-appraisal".
%
\chapter{Self-appraisal}

<This appendix should contain everything covered by the 'self-appraisal' criterion in the mark scheme. Although there is no length limit for this section, 2---4 pages will normally be sufficient. The format of this section is not prescribed, but you may like to organise your discussion into the following sections and subsections.>

\section{Critical self-evaluation}

Code could have been refactored better into a more modular structure with clearer responsiblities for each file (an attempt was made with the two javascript files making one handle the connection and the other to handle the board)

\section{Personal reflection and lessons learned}

How to use websockets to make real time applications
I stumbled across polling before knowing what it was

\section{Legal, social, ethical and professional issues}

<Refer to each of these issues in turn. If one or more is not relevant to your project, you should still explain {\em why} you think it was not relevant.>

\subsection{Legal issues}

\subsection{Social issues}

\subsection{Ethical issues}

\subsection{Professional issues}


%
% Any other appendices you wish to use should come after "Self-appraisal". You can have as many appendices as you like.
%
\chapter{External Material}
<This appendix should provide a brief record of materials used in the solution that are not the student's own work. Such materials might be pieces of codes made available from a research group/company or from the internet, datasets prepared by external users or any preliminary materials/drafts/notes provided by a supervisor. It should be clear what was used as ready-made components and what was developed as part of the project. This appendix should be included even if no external materials were used, in which case a statement to that effect is all that is required.>




%
% Other appendices can be added here following the same pattern as above.
%



\end{appendices}
